\documentclass{dscproblemset}

\usepackage[margin=1in]{geometry}

% change to \showsolntrue to show solutions
\showsolnfalse

% determine the exam's version
\def\version{Z}

\begin{document}

\examfrontpage{
    DSC 40B
}{
    Midterm 01
}{
    December 21, 2022
}{
    Instructions: do not cheat.
}

\nameoneverypage{}

\begin{probset}

    \begin{prob}
        This is the first problem in the homework.

        \begin{soln}
            This is the solution.
        \end{soln}

    \end{prob}

    \begin{prob}
        This is a problem with several subparts.

        \begin{subprobset}
            \begin{subprob}
                Part One.
            \end{subprob}

            \begin{subprob}
                Part Two.
            \end{subprob}
        \end{subprobset}

    \end{prob}

    \begin{prob}[(Optional descriptions can be provided)]
        Designated response areas can be drawn.

        \begin{responsebox}{2in}
            And solutions can be placed inside.
        \end{responsebox}
    \end{prob}

    \begin{prob}
        They can be inline response areas, too:

        First name: \inlineresponsebox[2in]{testing}

        Last name: \inlineresponsebox[2in]{this}
    \end{prob}

    \newpage

    \begin{prob}
        Multiple choice questions are supported.

        \begin{choices}
            \choice alpha
            \correctchoice beta
            \choice gamma
            \choice delta
        \end{choices}

        The choice bubble's shape can be changed to any valid
        TiKZ node shape, such as ``rectangle'':

        \begin{choices}[rectangle]
            \choice alpha
            \correctchoice beta
            \choice gamma
            \choice delta
        \end{choices}
    \end{prob}

    \begin{prob}
        As are True/False questions.

        \Tf{}

    \end{prob}

    % redefine the text that appears before the problem number
    \renewcommand{\progprobtext}{Coding Problem}

    \begin{progprob}
        This is a programming problem; it is numbered on a different counter.

        \bubble{test}
    \end{progprob}

\end{probset}

\scratchpage{}

\end{document}
